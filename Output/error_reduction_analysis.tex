\documentclass[11pt,a4paper]{amsart}

\usepackage{amsmath, amssymb}
\usepackage{dsfont}
\usepackage{geometry}
\usepackage{commath}
\usepackage{url}
\usepackage{enumerate}
\usepackage{graphicx}
\usepackage{mathtools}
\usepackage{todonotes}
%\setlength{\parindent}{0pt}
\geometry{%
  includeheadfoot,
  margin=2cm
}

\newtheorem*{problem}{Problem}
\newtheorem*{lemma}{Lemma}
\theoremstyle{definition}
\newtheorem*{answer}{Answer}

\newcommand{\inp}[2]{\langle #1, #2 \rangle}
\newcommand{\bbone}{\mathds{1}}
\DeclareMathOperator*{\diam}{diam}
\DeclareMathOperator*{\spann}{span}
\DeclareMathOperator*{\init}{init}
\DeclareMathOperator*{\ext}{ext}
\newcommand{\udlrarrow}{\mathrlap{\,{\scriptscriptstyle \updownarrow}}{\scriptscriptstyle \leftrightarrow}}

%\input{/Users/janwesterdiep/Dropbox/Commands.tex}
\newcommand{\T}{{\mathcal T}}
\newcommand{\R}{\mathbb{R}}

\date{\today}
\begin{document}

\title{First look at Spacetime}
\maketitle

Let me keep it short. We did a first run of the spacetime project solving the
heat equation on the unit square $\Omega = [0,1]^2$ on the time interval
$I := [0,1]$. The true solution is 
\[
  u(t,x,y) := (t^2 + 1) (1-x)x(1-y)y.
\]

Given some level $L$, we build $X_\delta$ as the double tree consisting of all
double nodes $\lambda = (\lambda_{\text{time}}, \lambda_{\text{space}})$ for
which $2|\lambda_{\text{time}}| + |\lambda_{\text{space}}| \leq L$ (a \emph{sparse grid}).
We then build $Y_\delta$ through the formula halfway page 6 of \texttt{followup.pdf}.
We use the system on the bottom of page 6, and compute the data integrals in the
right-hand side using Gauss quadrature of sufficient order to be exact.

Denote the $2 \times 2$ block-matrix with $M$. We use MINRES with initial guess $x_0$
equal to the right-hand side $b$ of the system, and continue until the relative residual norm is `small':
\[
  \frac{\|Mx_k - b\|_2}{\|b\|_2} \leq 10^{-5}.
\]
This gives us a discrete solution $(\mu_\delta, u_\delta)$ for which we can measure
\[
  E(t) := t \mapsto \|\gamma_t(u_\delta - u)\|_{L_2(\Omega)}
\]
by computing
\[
  E(t)^2 = \inp{u(t)}{u(t)}_{L_2(\Omega)} + \inp{u_\delta(t)}{u_\delta(t)}_{L_2(\Omega)} - 2 \inp{u(t)}{u_\delta(t)}_{L_2(\Omega)}
\]
where we have the first term analytically, the second number can be computed by
an applicator of the space-time mass matrix, and the final term is approximated
using quadrature of sufficient degree. We will use this as a proxy for the error
$\|u - u_\delta\|_X$, which is difficult to measure.

Given $X_\delta$, we distinguish the dimension $\dim X_\delta$ of the function space
and $\# X_\delta$, being the total number of double nodes in the double tree.

\section*{Convergence rate}
The code right now seems to solve the problem, as can be seen by Figure~\ref{fig:time-slice}:
Increasing $L$ (thereby increasing $\dim X_\delta$, yields an error reduction.
We discern a pre-asymptotic regime: the error hardly decreases between levels $L=1$
and $2$. Moreover, errors seem to taper off near the final levels $L$, which may
have something to do with our solver tolerance.

In fact, the errors decay at an algebraic rate somewhere between $0.75$ and $0.85$,
depending on the time $t$ at which the errors are measured, as well as whether or
not we keep the full range of errors when calculating the rate (for instance, in
the brown line, we slice away level $1$ because it is in the pre-asymptotic regime,
and levels $11$ and $12$ because the convergence stagnates in this regime).
\begin{figure}[h!]
  \includegraphics[width=0.495\linewidth]{error_reduction_time_slices.pdf}
  \includegraphics[width=0.495\linewidth]{error_reduction_rates.pdf}
  \caption{Left: Time slice errors. Right: rates.}
  \label{fig:time-slice}
\end{figure}

\section*{Computational cost}
Let's measure the time it takes to perform a single application of the system
matrix. We do not take into account the setup cost (like
computing $\Sigma$ and $\Theta$). Figure~\ref{fig:linearity} paints a slightly
demotivating picture, in that the bottom graph is not constant. We do have to
place an asterisk here: for small systems, the produced numbers are averaged over
a small number of iterations, so they change a lot from run to run.
\begin{figure}[h!]
  \includegraphics[width=0.6\linewidth]{error_reduction_linearity.pdf}
  \caption{Top: time (s) per apply; bottom: time (ms) per apply per DoF.}
  \label{fig:linearity}
\end{figure}

\section*{The solver}
As mentioned before, we use MINRES with default parameters (tolerance $10^{-5}$,
initial guess $x_0$ equal to the RHS $b$). In Figure~\ref{fig:minres}, we show
the \emph{absolute} residual norm $\|M x_k - b\|_2$ as a function of $k$ for a
selection of solves.
\begin{figure}[h!]
  \includegraphics[width=0.495\linewidth]{error_reduction_minres_history_semilogy.pdf}
  \includegraphics[width=0.495\linewidth]{error_reduction_minres_history_loglog.pdf}
  \caption{Left: semi-log-y MINRES convergence graph for some solves. Right: Same results in log-log format.}
  \label{fig:minres}
\end{figure}
We see that the number of iterations increases strongly, suggesting that we could
definitely use a good preconditioner.


\end{document}

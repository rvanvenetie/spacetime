\documentclass[12pt]{amsart}
\usepackage[T1]{fontenc}
\usepackage{amsmath}

\title{Space-time adaptivity for parabolic evolution equations}
\author{Rob Stevenson \and Raymond van Veneti\"e \and Jan Westerdiep$^\star$}
\address{$^\star$ Korteweg-de Vries Institute for Mathematics, University of Amsterdam}
\email[Jan Westerdiep]{j.h.westerdiep@uva.nl}

\begin{document}

\maketitle

\section*{ABSTRACT}


Taking the well-posed mixed simultaneous space-time variational formulation
introduced in~\cite{And13}, we use methods previously developed in~\cite{SW20}
to construct an adaptive refinement loop that produces nested space-time
approximations as linear combinations of tensor-products of wavelets in time and
finite elements in space.

The sequence of approximations produced by this loop exhibits optimal convergence
rate in the number of degrees of freedom, at an optimal computational cost. The
result is that for certain problems, large-scale parallel space-time solvers may
be unnecessary and can be replaced by a method that adapts itself to the problem.

\begin{thebibliography}{99}

\bibitem[And13]{And13}
R.~Andreev.
\newblock Stability of sparse space-time finite element discretizations of
 linear parabolic evolution equations.
\newblock {\em IMA J. Numer. Anal.}, 33(1):242--260, 2013.

\bibitem[SW20]{SW20}
\newblock Stability of Galerkin discretizations of a mixed space-time variational
   formulation of parabolic evolution equations.
\newblock {\em IMA J. Numer. Anal.}, 2020.

\end{thebibliography}


\end{document}
